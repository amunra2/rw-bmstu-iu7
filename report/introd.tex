\chapter*{Введение}
\addcontentsline{toc}{chapter}{Введение}

При появлении первых компьютерных машин, люди задумались о возможности автоматизировать написание картин и музыки. Поскольку сочинение музыки является долгим процессом, потому что композитор проводит тяжелую работу, вопрос генерации музыки крайне важен и по сей день. Роль музыки в мире невозможно переоценить -- практически в каждой игре, в каждом магазине, в каждом фильме используется какая-нибудь мелодия. Именно поэтому важно иметь возможность использовать синтетически сгенерированные композиции в определенных местах.

Целью данной работы является классификация современных методов генерации музыки полифонического склада. В рамках работы необходимо решить следующие задачи:

\begin{itemize}
    \item рассмотреть виды генерирумых музыкальных фрагментов;
    \item выделить основные проблемы и сформулировать ограничения предметной области при генерации музыкального произведения;
    \item описать методы генерации музыки;
    \item провести сравнение методов на основании критерия оценки;
    \item выбрать алгоритм, который наиболее подходит к сформулированным ограничениям предметной области.
\end{itemize}






% Научно-исследовательска работа дает возможность подробнее разобраться в теме будущей курсовой или дипломной работы.


% Далее будут приведены тема, ее суть, а также статьи, которые были выбраны для её исследования.


% \textbf{Тема:} \textit{Метод генерации пьесы полифонического склада на основе формальных грамматик.}

% \textbf{Суть:} Исследовать данный метод -- описать грамматику и ее правила, познакомиться с алгоритмами, которые применяются в компьютерной лингвистике

% Статьи для ознакомления с темой.

% \begin{enumerate}
%     \item Полифония при генерации музыки: \cite{1-polyphonic}, \cite{2-polyphonic}, \cite{3-polyphonic}, \cite{4-polyphonic}.
%     \item Методы генерации текста (используются схожие алгоритмы, которые используются при генерации музыки): \cite{1-text}, \cite{2-text}, \cite{3-text}, \cite{4-text}, \cite{5-text}.
%     \item Фуга, как один из лучших жанров музыки для её генерации: \cite{1-fugue}, \cite{2-fugue}, \cite{3-fugue}.
%     \item Использование теории языков и грамматик в программировании: \cite{1-grammar}, \cite{2-grammar}, \cite{3-grammar}.
%     \item Различные методы генерации музыки: \cite{1-music}, \cite{2-music}, \cite{3-music}, \cite{4-music}, \cite{5-music}, \cite{6-music}.
%     \item Использование теории языков и грамматик при генерации музыки: \cite{1-grammusic}, \cite{2-grammusic}.
% \end{enumerate}
