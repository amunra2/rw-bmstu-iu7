\chapter{Генерация музыки}

Далее будут приведены тема, ее суть, а также статьи, которые были выбраны для её исследования.

\section{Формулировка темы}

\textbf{Тема:} \textit{Метод генерации пьесы полифонического склада на основе формальных грамматик.}

\textbf{Суть:} Исследовать данный метод -- описать грамматику и ее правила, познакомиться с алгоритмами, которые применяются в компьютерной лингвистике

\section{Исследование темы}

Статьи для ознакомления с темой.

\begin{enumerate}
    \item Полифония при генерации музыки: \cite{1-polyphonic}, \cite{2-polyphonic}, \cite{3-polyphonic}, \cite{4-polyphonic}.
    \item Методы генерации текста (используются схожие алгоритмы, которые используются при генерации музыки): \cite{1-text}, \cite{2-text}, \cite{3-text}, \cite{4-text}, \cite{5-text}.
    \item Фуга, как один из лучших жанров музыки для её генерации: \cite{1-fugue}, \cite{2-fugue}, \cite{3-fugue}.
    \item Использование теории языков и грамматик в программировании: \cite{1-grammar}, \cite{2-grammar}, \cite{3-grammar}.
    \item Различные методы генерации музыки: \cite{1-music}, \cite{2-music}, \cite{3-music}, \cite{4-music}, \cite{5-music}, \cite{6-music}.
    \item Использование теории языков и грамматик при генерации музыки: \cite{1-grammusic}, \cite{2-grammusic}.
\end{enumerate}