\chapter*{Заключение}
\addcontentsline{toc}{chapter}{Заключение}

В ходе данной работы были классифицированы некоторые алгоритмы генерации музыкального фрагмента с использованием современных технологий. В качестве самого точного алгоритма для дальнейшего исследования был выбран алгоритм, позволяющий генерировать музыку полифонического склада на основе контекстно-зависимых грамматик, поскольку он на основе статистического опроса оказался самым точным.

Цель, которая была поставлена в начале работы была достигнута, а также в ходе выполнения научно-исследовательской работы были решены следующие задачи:

\begin{itemize}
	\item были описаны виды проблем, возникающих при генерации музыки;
    \item были описаны методы, которые позволяют получить музыкальный фрагмент;
	\item проведен сравнительный анализ по выделенному критерию для всех методов генерации;
	\item выбран алгоритм, который наиболее подходит к сформулированным ограничениям предметной области.
\end{itemize}